\part{Chapitre 2}
\section{Définitions}
\subsection{Les charges élémentaires}
Les charges élémentaires sont les électrons et les protons:
\begin{equation}
	e^- = 1,6.10^{-19}C\\
\end{equation}
\begin{equation}
	p^+ = 1,6.10^{-19}C\\
\end{equation}
\begin{equation}
	1C = 6,25.10^{18}e^-
\end{equation}
 \subsection{La permitivité du vide}

 \begin{equation}
	 \epsilon_{0} = \frac{1}{36.\pi}.10^{-9} \quad \textrm{en} \quad \frac{[F]}{[m]}
 \end{equation}

 \subsection{Le champ électrique}
 Le champ électrique est la force qui s'exerce sur une charge positive unitaire.
 \begin{equation}
	 \overrightarrow{E} = \frac{\overrightarrow{F}}{q} \quad \textrm{en} \quad \frac{[N]}{[C]} \quad \textrm{ou en} \quad \frac{[V]}{[m]}
 \end{equation}

 \subsection{La différence potentiel}
 \begin{equation}
	 U_{p} = \frac{1}{4.\pi.\epsilon_{0}}.\sum\frac{|q_{i}|}{r_{i}}
 \end{equation}

 Avec:
 \begin{itemize}
	 \item \textbf{U\textsubscript{p}}: Le potentiel au point p en \textbf{Volt} (\textit{V})
	 \item \textbf{q\textsubscript{i}}: Les charges ponctuelles i \textbf{Coulomb} (\textit{C})
	 \item \textbf{r\textsubscript{i}}: Les distances entre p et les q\textsubscript{i} en \textbf{Coulomb} (\textit{C})
 \end{itemize}

 \subsection{Farad}

 \begin{equation}
	 C = \frac{q_{A}}{U_{AB}} =\frac{q_{B}}{U_{AB}}
 \end{equation}
 \begin{equation}
	 [F] = \frac{[C]}{[V]}
 \end{equation}
 Avec:
 \begin{itemize}
	 \item \textbf{C}: La capacité du condensateur en \textbf{Farad} (\textit{F})
	 \item \textbf{q\textsubscript{A/B}}: La charge de l'armature A/B en \textbf{Coulomb} (\textit{C})
	 \item \textbf{U\textsubscript{AB}}: La différence de potentiel entre les deux armatures en \textbf{Volt} (\textit{V})
 \end{itemize}

 Dans le vide:
 \begin{equation}
	 C = \epsilon_{0}.\frac{S}{d}
 \end{equation}
 Dans un autre isolant:
 \begin{equation}
	 C = \epsilon_{0}\epsilon_{r}.\frac{S}{d}
 \end{equation}
 \begin{equation}
	 [F] = [F/m].\frac{[m^2]}{[m]}
 \end{equation}
 Avec:
 \begin{itemize}
	 \item \textbf{C}: La capacité du condensateur en \textbf{Farad} (\textit{F})
	 \item \textbf{S}: La surface des armatures en \textbf{mètre carré} (\textit{m\textsuperscript{2}})
	 \item \textbf{d}: La distance entre les armatures en \textbf{mètre} (\textit{m})
	 \item \textbf{$\epsilon\textsubscript{r}$}: La permitivité de l'isolant (\textit{F/m})

 \end{itemize}



 \section{Lois et formules importantes}

 \subsection{Loi de Coulomb}
 \begin{equation}
	 \overrightarrow{F_{q_{2}}} = \frac{|q_{1}|.|q_{2}|.\overrightarrow{u}}{4.\pi.\epsilon_{0}.r^2}
 \end{equation}
 \begin{equation}
	 |F_{q_{2}}| = -|F_{q_{1}}|
 \end{equation}
 Avec:
 \begin{itemize}
	 \item \textbf{q\textsubscript{1}}: La charge ponctuelle 1 positive en \textbf{Coulomb} (\textit{C})
	 \item \textbf{q\textsubscript{2}}: La charge ponctuelle 2 positive en \textbf{Coulomb} (\textit{C})
	 \item \textbf{r}: La distance entre  q\textsubscript{1} et q\textsubscript{2} en \textbf{mètre} (\textit{m})
 \end{itemize}

 \begin{equation}
	 |\overrightarrow{E}| = \frac{|q_{1}|.\overrightarrow{u}}{{4.\pi.\epsilon_{0}.r^2}}
 \end{equation}

 \subsection{Groupements de condensateurs}
 \subsubsection{En parallèle}

 \begin{equation}
	 C_{eq} = \sum C_{i}
 \end{equation}
 \subsubsection{En série}

 \begin{equation}
	 C_{eq} = \frac{1}{\sum \frac{1}{C_{i}}}
 \end{equation}


 \subsection{L'énergie électrostatique}
 \begin{equation}
	 W = \frac{1}{2}.C.U^2
 \end{equation}
 \begin{equation}
	 [J] = [F].{[V]}^2
 \end{equation}
