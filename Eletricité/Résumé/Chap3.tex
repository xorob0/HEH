\part{Chapitre 3}
 \section{Lois et formules importantes}
 \subsection{Théorème de Thévenin}
Un réseau électrique linéaire vu de deux points est équivalent à un générateur de tension parfait dont la force électromotrice est égale à la différence de potentiels à vide entre ces deux points, en série avec une résistance égale à celle que l'on mesure entre les deux points lorsque les générateurs indépendants sont rendus passifs.\\
\hyperref[Vidéo d'exemple]{https://www.youtube.com/watch?v=cSiJ08XExAE}
\subsection{Théorème de Norton}
Tout circuit linéaire est équivalent à une source de courant idéale I, en parallèle avec une simple résistance R.
\begin{itemize}
	\item Le courant de Norton est le courant entre les bornes de la charge lorsque celle-ci est court-circuitée, d'où Ic = I (court-circuit).
	\item La résistance de Norton est celle mesurée entre les bornes de la charge lorsque toutes les sources sont rendues inactives, en court-circuitant les sources de tension et en débranchant les sources de courant.\\
\end{itemize}
\hyperref[Vidéo d'exemple]{https://www.youtube.com/watch?v=sCx-uHiobb4}
